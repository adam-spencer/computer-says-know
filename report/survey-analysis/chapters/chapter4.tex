\chapter{Progress, Conclusions, and Project Plan}\label{ch:conclusions-and-proj-plan}

\section{Progress}\label{sec:progress}

To this date, the progress on this work is as follows;

\begin{itemize}
    \item Access has been granted for this project to utilise Sheffield's High Performance
    Computing (HPC) `Bessemer' cluster, enabling ASR models to be run over entire speech corpora.
    \item Whisper\cite{whisper} has been installed on Bessemer and tested using the pre-trained
    `large-v2' model.
    \item The Switchboard I (LDC97S62)\cite{switchboard-ldc} and MACROPHONE
    (LDC94S21)\cite{macrophone} corpora have been added to Bessemer as they were found to
    contain reasonably diverse age ranges and are readily available via the LDC\@.
    \item The LDC stores their data in NIST SPHERE file format, meaning they contain both
    metadata and audio data.
    Whisper can't read this file format so various scripts have been written to handle the data
    and process it into WAVE format en-masse to enable batch transcription tasks.
    \item For the purposes of this project, access has been granted to the LifeLUCID
    corpus~\cite{lifelucid} by the rights owner, though a delay due to the Christmas holiday has
    meant these data have yet to be copied onto Bessemer.
    \item Whisper has been used to successfully transcribe part of the MACROPHONE corpus with the
    purpose of gaining an understanding of the way metadata is stored in LDC corpora, providing an
    approximation of Whisper's runtime, and exploring the HPC system's functionality.
\end{itemize}

\section{Conclusion}\label{sec:conclusion}

This report has provided the path for the remainder of the project to follow.
A review of the literature on age-related speech and how this affects ASR systems has
provided the information necessary to refine the scope and direction to produce a full
exploration of how ASR could be improved to better understand the elderly.

\section{Project Plan}\label{sec:proj-plan}

To allow for enhanced flexibility, the plan is split into five ordered chunks;
each should last approximately two weeks but are intended to be adaptable.

\subsection{Chunk 1}\label{subsec:wk1}

\begin{itemize}
    \item Move LifeLUCID onto Bessemer.
    \item Install wav2vec 2.0 on Bessemer.
    \item Experiment with wav2vec to gain familiarity.
    \item Improve efficiency of batch transcription tasks by reading over the HPC literature.
    \item Write Python scripts to enable WER \& PER calculation from ASR-generated transcripts.
    \item Search for further corpora and open-source ASR systems relevant to the project.
\end{itemize}

\subsection{Chunk 2}\label{subsec:wk2}

\begin{itemize}
    \item Review literature to gain a less ASR-focused understanding of speech irregularities and
    how they relate to age.
    Use spectral analysis and other signal processing techniques to produce examples of these
    characteristics.
    \item Collect results to compare existing ASR systems across corpora with a focus on
    age-related performance.
    \item Present an analysis of these results.
\end{itemize}

\subsection{Chunk 3}\label{subsec:wk3}

\begin{itemize}
    \item Research the adaptation and fine-tuning of ASR models.
    \item Experimentally fine-tune existing models.
    \item Compare the results of fine-tuned models to the literature.
\end{itemize}

\subsection{Chunk 4}\label{subsec:wk4}

\begin{itemize}
    \item Suggest age-related areas of improvements for ASR\@.
    \item Write a detailed analysis of findings.
\end{itemize}

\subsection{Chunk 5}\label{subsec:wk5}

\begin{itemize}
    \item Prepare presentation of findings.
    \item Finalise report formatting and quality of writing.
\end{itemize}