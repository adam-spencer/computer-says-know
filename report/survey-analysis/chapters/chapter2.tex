\chapter{Literature Survey}\label{ch:literature-survey}

\section{Modern ASR}\label{sec:modern-asr}

The results of ASR models are reported as a percentage word error rate (WER), meaning the
percentage of words produced by the model which are incorrect compared to the real transcript (i
.e. lower WER means higher accuracy).\\

In late September 2022, the OpenAI research laboratory (known for their `GPT-3' language model)
released a new open-source ASR system known as `Whisper'~\cite{whisper}.
Whisper is unique in being very large (trained on 680,000 hours of speech data), open-source, and
fully supervised;
meaning all the training data used to create the model has been accurately labeled and
quality-checked by humans, unlike the much larger unsupervised `BigSSL' model (1,000,000+ hours
of data)~\cite{bigssl}.

Their accuracy results are promising across a range of different speech corpora, outperforming
previous state-of-the-art \emph{wav2vec 2.0}~\cite{wav2vec}.
Interestingly, Whisper doesn't achieve higher performance than some other models on specific
corpora, for example on \emph{LibriSpeech test-other}~\cite{librispeech} it is outperformed by
two models built atop \emph{wav2vec}~\cite{zhang2020,chung2021}, though across a more diverse set
of speech corpora Whisper achieves much higher lower WER~\cite{whisper}.\\

\section{Elderly Speech}\label{sec:elderly-speech}



\section{Data Collection}\label{sec:data-collection}

The key requirement for comparing these systems is a large dataset of speech recordings over a
wide range of ages, with labels denoting the age of the speaker and an accurate transcript to
compare ASR performance against.
There is a surprising lack of relevant resources online, with most speech corpora featuring
labeled speaker age not featuring a diverse age range.

The LifeLUCID Corpus is, however, collected with the express purpose of providing a wide range of
speaker ages~\cite{lifelucid}.
This resource has great potential in providing an insight into the age-variable performance of
ASR because it features 104 British English speakers between 8 and 85 years of age.

It is of interest to observe any patterns in errors made by each system, especially if there are
any similarities in where they lose accuracy.
Once the areas of error are understood, comparing and contrasting audio spectra may allow for
insight into which features of speech create problems for ASR .

