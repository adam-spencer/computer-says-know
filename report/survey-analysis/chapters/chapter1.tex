\chapter{Introduction}\label{ch:introduction}

Automatic Speech Recognition systems are rapidly increasing in accuracy, from 17.1\% word error
rate (WER) in 2011~\cite{seide2011} to 5.1\% in 2021~\cite{Ng2021} on the same Switchboard
corpus~\cite{switchboard}.
Though this increase in accuracy is promising for the average English speaker, elderly speakers
have different speech patterns~\cite{Horton2010} and are transcribed less accurately than other age
groups~\cite{picone1990}.

The objective of this work is to provide an in-depth understanding of how the differences
between elderly and non-elderly speech influence the accuracy of different types of ASR systems,
and which (if any) current models best fit the task of transcribing elderly speech.

The Clarity Project is an EPSRC-funded research project aiming to improve hearing aid signal
processing.
Through the NHS, they have collected a large quantity of recordings of elderly speakers
completing an intelligibility test in which they are required to repeat speech played to them
through headphones.
Evaluating such a test is deceptively simple;
the content of the subjects' response must be compared to the original audio recording to observe
how well the subject understood what they heard.

Problems arise when transcribing these recordings because the elderly participants' speech is not
well understood by existing ASR systems.

\section{Aims and Objectives}\label{sec:aims-and-objectives}

The aims and objectives of my project are as follows;

\begin{itemize}
    \item Understand the relationship between age and speech intelligibility
    \item Analyse the effect of age on the accuracy of ASR systems
    \item Compare the performance of different methods of ASR to determine what features of an
    ASR system are better suited to understanding elderly speakers
    \item
\end{itemize}


\section{Overview of the Report}\label{sec:overview-of-the-report}

