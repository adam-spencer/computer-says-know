\chapter{Requirements and Analysis}\label{ch:requirements-and-analysis}

\section{Aims and Objectives}\label{sec:aims-and-objectives2}

The following subsections shall detail the aims and objectives of this dissertation project.

\subsection{Demonstrate Age-Related Deviations in Speech}\label{subsec:aim1}

As presented in the literature review, age is understood to have an impact on the structure of
speech, thus an aim of this work is to better visualise and collate this understanding using
frequency spectra and searching for deviations in speech across multiple large speech corpora.

An example experiment to visualise age-related deviations in speech may consist of using ASR to
isolate occurrences of the same word across speech corpora, then comparing frequency spectra
across recordings.

\subsection{Compare the Effectiveness of Different ASR Models on Elderly Speech}\label{subsec:aim2}

This objective requires either a series of ASR systems to use to obtain experimental results, or
instead highly detailed, full transcription results across common speech corpora which contain
elderly speakers.
Accessing these resources may be difficult because many ASR systems aren't available to be used
freely and the full results present in their literature are often unpublished.

\subsection{Experimentally Alter ASR Models}\label{subsec:aim3}

In~\ref{subsec:bespoke-systems}, the potential benefit of altering an existing ASR system was
discussed.
Though it may be difficult, the potential benefit to the work of the Clarity project means that
this is a relevant and important objective, at least so far as performing experiments to test the
viability of such a method.

\subsection{Produce Suggestions to Improve ASR Performance for Elderly Speakers}\label{subsec:aim4}

Using findings from the previous objectives, this work aims to provide suggestions to those
interested in the furthering of ASR technology as it relates to transcribing elderly and less
common speech patterns.
This objective is very important because there is a clear disparity in the accuracy of ASR
transcriptions for elderly speakers and a distinct lack of interest from contemporary researchers
made apparent by this disparity.