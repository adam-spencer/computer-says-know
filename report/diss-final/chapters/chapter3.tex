\chapter{Requirements and Analysis}\label{ch:req-and-analysis}

The objective of this work is not to produce a fully-working, infallible system which aims to receive actual use by transcribers, rather, the aim of this work is to explore the current state of the field of ASR and to understand the extent that current ASR technology could provide aid to a human transcriber.
With the purpose of facilitating an extensive evaluation, this chapter shall list the requirements for this work to meet its objective and provide a detailed analysis of each requirement.

\section{Understand the motivations of computer-aided transcription}

\mycomment{
  TODO:
  * Rewrite / expand on this
  * define a 'target user'
  * actually motivate semi auto transcription!
}
Before exploring how a computer system may aid a human transcriber, it is important to understand;

\begin{itemize}
        \item \emph{why} a human transcriber may require aid;
        \item to \emph{whom} a computer-aided transcription system would provide benefit; and
        \item what the \emph{extent} of such a benefit would be.
\end{itemize}

If this report does not properly motivate computer-aided transcription to a reader, it will have partially failed in its purpose 

\section{Generate transcripts using ASR}\label{sec:generate-transcripts}

Evaluating the quality of ASR transcription requires a key set of data; ASR-generated transcripts.
Rather than comparing different ASR systems, Whisper\cite{whisper} has been chosen as the only system to use for generating transcripts because;

\begin{itemize}
        \item it is new (made available in September 2022);
        \item it is entirely free and open-source, meaning it is easily modifiable and available to be used without licence; and
        \item it reportedly achieves very good results across different speech corpora.
\end{itemize}

As mentioned in the literature review, Whisper is implemented using \emph{PyTorch}, meaning this work would benefit greatly from access to high-performance GPUs.
This would enable fast turnaround times when transcribing large speech corpora and thus enable rapid evaluation and tweaking of settings to minimise erronious results.

The key to generating useful transcripts is some high-quality speech recordings from a speech corpus.
While preliminary testing of Whisper may use data from any available corpora, it would be very useful to obtain some data which is;

\begin{itemize}
        \item not present in Whisper's training data, to prevent the model from regurgitating labels for data it has already seen;
        \item is well-suited to Whisper's particularities, aiming to maximise the usefulness of results; and
        \item is representative of real data which would benefit from computer-aided transcription, as to enable more practical evaluation.
\end{itemize}

Two preliminary aims, therefore, are to understand what kind of data is suited to Whisper and then what kind of data would benefit from computer-aided transription.
Once these aims are understood, a suitable dataset may be gathered and used for evaluation, however it is also useful to understand the caveats related to using well-suited data!
The na\"{i}ve assumption that all data seen by \emph{any} computer system is 'perfect' would misrepresent the usefulness of the system in question.
To combat this, this work must properly acknowledge the limited extent to which a computer-aided transcription system using Whisper is viable, and evaluate how the viability could be increased to be more applicable to real-world tasks.

\section{Implement various confidence measures}

Neural network confidence is widely discussed in the literature.
Considering the aim of this work, it shall focus on re-creating and applying existing measures of confidence to Whisper, whether through modification to the model itself or through inference of the model's output.

If some such modifications fall out of scope of the project, this work would still benefit from a discussion of how those approaches may be applied to a future system and what advantages they may bring.

\section{Understand the effectiveness of selected confidence measures}

Evaluation of the extent to which Whisper may aid human transcription may be done by comparing the accuracy of Whisper's predictions against a reference and the reported model confidence.

This type of evaluation requires taking the following steps to complete;

\begin{enumerate}
        \item Selection of suitable measure(s) of system accuracy
        \item Extensive normalisation of text to facilitate accurate measurements
        \item Visualisation of results
\end{enumerate}

The standard across surveyed literature is \emph{word error rate} (WER), despite potential limitations.
For the sake of evaluation, other measurements such as \emph{phone error rate} (PER) and \emph{character error rate} (CER) should be calculated alongside WER.

\section{Explore designs for computer-aided transcription}

It would be of great utility to understand how system confidence may aid a human transcriber as this would further refine the 'lens' through which the system may be evaluated, and as such is vital to the completion of this work.
There's limited use in a purely theoretical exploration of a computer system such as this, which is designed to be interfaced with by a human.
Instead, demonstrating the benefits of a computer-aided transcription system would be easily facilitated using a graphical program.

A number of considerations are required for the design of this system specifically due to its intended nature to serve as an example rather than a final implementation, including;

\begin{itemize}
        \item several design iterations should be produced to demonstrate different extents of computer aid;
        \item each design decision should be discussed thoroughly to demonstrate the intended effect and any noteable caveats; and
        \item a suitable number of screenshots are required in the appendix to demonstrate use of the system.
\end{itemize}

\section{Evaluation}

\section{Ethical, Professional and Legal Issues}

