\chapter{Requirements and Analysis}\label{ch:req-and-analysis}

\section{Project Requirements}\label{sec:requirements}

The objective of this work is not to present a new paradigm through which transcription may develop, nor is it to produce a fully-working, infallible system which aims to receive actual use by transcribers.
Rather, the aim of this work is to explore the current state of the field of ASR and to understand the extent that current ASR technology could provide aid to a human transcriber.

To further understand its objective, the requirements of this work are as presented in the following subsections.

\subsection{Motivate computer-aided transcription}

Before exploring how a computer system may aid a human transcriber, it is important to understand;

\begin{itemize}
        \item \emph{why} a human transcriber may require aid;
        \item to \emph{whom} a computer-aided transcription system would provide benefit; and
        \item what the \emph{extent} of such a benefit would be.
\end{itemize}

\subsection{Generate transcripts using ASR}

Evaluating the quality of ASR transcription requires a key set of data; ASR-generated transcripts.
Rather than comparing different ASR systems, Whisper\cite{whisper} has been chosen as the only system to use for generating transcripts because;

\begin{itemize}
        \item it is new (made available in September 2022);
        \item it is entirely free and open-source, meaning it is easily modifiable and available to be used without licence; and
        \item it reportedly achieves very good results across different speech corpora.
\end{itemize}

Whisper\cite{whisper} is implemented in Python using the PyTorch\cite{pytorch} library which allows computation to take place on GPUs which support CUDA, meaning transcripts can be generated very quickly on a high-performance computer (HPC) system.
Luckily, the University of Sheffield offers access to HPC clusters\cite{shef-hpc}, and I have been granted access to the \emph{Bessemer} cluster for the completion of this project.

The key to generating useful transcripts is some high-quality speech recordings from a speech corpus.
While preliminary testing of Whisper may use data from any available corpora, it would be very useful to obtain some data which is;

\begin{itemize}
        \item not present in Whisper's training data, as to prevent the model from regurgitating labels for data it has already seen; and
        \item is well suited to the task of computer-aided transcription, as to enable more practical evaluation.
\end{itemize}

It is also useful, however, to understand the caveats related to using well-suited data!
The na\"{i}ve assumption that all data seen by \emph{any} computer system is 'perfect' would misrepresent the usefulness of the system in question.
To combat this, this work must acknowledge the limited extent to which a computer-aided transcription system using Whisper is viable, and evaluate how the viability could be increased to be more applicable to real-world tasks.

\subsection{Experiment with confidence measures}

Neural network confidence is widely discussed in the literature.
Rather than attempt to create a novel method for calculating a system's confidence, this work should focus on re-creating and applying existing measures of confidence to Whisper, whether through modification to the model itself or through inference of the model's output.

It would be of great utility to understand how system confidence may aid a human transcriber, as well as how exactly 'confidence' shall be defined in the case of this specific task.
Such understandings would further refine the 'lens' through which the system may be evaluated, and as such are vital to the completion of this work.

\subsection{Explore designs for computer-aided transcription}

There's limited use in a purely theoretical exploration of a computer system which is designed to be interfaced with by a human, such as this.
Providing various design concepts for a real computer system shall aid the reader in grasping the benefit of a 

\section{Analysis}

This section shall provide analyses of the requirements cited in \ref{sec:requirements}, presenting an analysis of each aforementioned requirement in corresponding order to that in which the requirements are given.
The general intention of this section is to determine the lens through which the project shall be evaluated, and to contribute a detailed motivation for the decisions made in both the design and implementation of the project.

The following subsections constitue an analysis of the aims of the project.

\subsection{}
\subsection{}
\subsection{}
\subsection{}

\section{Ethical, Professional and Legal Issues}\label{sec:ethics-etc}

