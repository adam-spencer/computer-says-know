\chapter{Implementation and Testing}\label{ch:implementation-and-testing}

\section{Preparing the Data}

\mycomment{
  This needs rewriting! TODO:

  * Section about HPC/Bessemer specs and runtime

  }
\subsection{Generating Utterances}

The contents, beginning, and end of every utterance where computed using the data available in the \emph{TextGrid} files using a Python script named \texttt{get\_utterances}.
This script operates over a directory containing \emph{TextGrid} files, writing out the utterances as files in \emph{JSON} format.

The script also takes as args; a minimum time between tokens required to end the utterance and a maximum pause time allowed within one utterance.
These thresholds allow utterances to be fine-tuned by a user, leading to fewer drawn-out or unreasonably short utterances.

\emph{JSON} was selected due to its ability to be easily read and understood by a human, unlike \emph{TextGrids}.
This allowed for simple verification of the data without the need for more specific software to view the files.

\subsection{Audio Segmentation}

Another Python script named \texttt{segment\_audio} was created to generate audio files for each utterance. 
Given two directories as input; one containing \texttt{.json} files (as output by the \texttt{get\_utterances} script) and the other containing \texttt{.wav} files representing each audio recording, the audio is split along the beginning and end times of each utterance and output to a new directory.

This script uses the \emph{python-soundfile} module\cite{pysoundfile} to load audio files into \emph{NumPy}\cite{numpy} arrays.
By multiplying the sampling rate of the audio by the start- and end-times of each utterance, the array indices at the start and end of each utterance are computed.
Array slices between these indices represent each utterance, which can then be saved to new audio files using the \emph{python-soundfile} module.

\section{ASR With Whisper}

Whisper is available as a Python module named \texttt{whisper}\cite{pypi-whis}.
The module features a \texttt{transcribe()} function to transcribe audio files given as a parameter to the function and return an object containing the output of Whisper.

\mycomment{explain whisper output}
\mycomment{explain why I made a script to run it}
