\chapter{Introduction}\label{ch:introduction}

Automatic Speech Recognition (ASR) is difficult.
Some 'state-of-the-art' systems claim to produce error rates as low as 4\%\cite{wav2vec2} while those same models achieve errors as high as 40\% on other datasets\cite{szymanski2020we}.
Clearly modern ASR systems aren't quite ready to be used to transcribe speech with an expectation of highly-accurate results.

On the other hand, services which employ humans to produce speech transcripts such as \emph{Rev}\cite{rev} charge \$1.50 per minute and take around 24 hours to complete, though they claim error rates of approximately 1\%.
For those without access to substantial funding or time to wait, human transcripts are out of the question.

Rather than rely on either humans or ASR to produce a high quality transcript, why not leverage the speed of ASR and the accuracy of a human?
If the ASR system were able to estimate its \emph{confidence} in its output, a human transcriber would only need to correct the results which are more likely to contain errors and work their way through the computer-generated results until the frequency of errors had reduced to a level they could accept.
The problem is, Whisper doesn't report its confidence in the transcript it produces.

This work shall explore the applicability of a new free and open-source ASR model called '\emph{Whisper}'\cite{whisper} to be used to aid a human in producing high-quality transcripts.
By exploring ways to derive model confidence, some metrics to rank results in order of expected correctness shall be implemented, followed by a demonstration of different techniques using ASR confidence to aid a human transcriber.
Finally, the ability of different confidence metrics to accurately predict transcription errors will be tested and discussed in detail.

The following chapters shall review and analyse the literature surrounding ASR and confidence, build some requirements for this project, design a system to find ASR confidence from a speech corpus and finally test the quality of this system using a rudimentary simulation.
